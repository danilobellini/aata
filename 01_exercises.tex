\documentclass{article}[10pt]

\usepackage[T1]{fontenc}
\usepackage{amsmath} % \operatorname
\usepackage{amssymb} % \blacksquare
\usepackage{dsfont} % \mathds
\usepackage{array}
\usepackage{enumerate}
\usepackage[a4paper, margin=2cm, bottom=3cm]{geometry}
\usepackage{cancel}

\setlength{\parindent}{0cm} % Don't indent
\everymath{\displaystyle}

\newenvironment{exercise}[1]
    {\noindent\rule{2cm}{0.4pt} \\
     \textbf{#1.}}
    {}

\newcommand{\answer}{

  \underline{Answer}

}

\newcommand{\qed}{

\hfill\ensuremath{\blacksquare}

}

\begin{document}

\begin{center}
  {\Large Thomas W. Judson's
          Abstract Algebra~--- Theory and Applications} \\
  {\large Exercise solutions by Danilo J. S. Bellini} \\
\end{center}


\setcounter{section}{1}\setcounter{subsection}{2}
\subsection{Exercises}


\begin{exercise}{1}
  Suppose that
  \[
    \begin{array}{>{\displaystyle}r
                  >{\displaystyle}c
                  >{\displaystyle}l}
      A & = & \left\{ x : x \in \mathds{N} \operatorname{and}
                          x \;\text{is even} \right\}, \\
      B & = & \left\{ x : x \in \mathds{N} \operatorname{and}
                          x \;\text{is prime} \right\}, \\
      C & = & \left\{ x : x \in \mathds{N} \operatorname{and}
                          x \;\text{is a multiple of}\; 5 \right\}. \\
    \end{array}
  \]
  Describe each of the following sets.
  \begin{enumerate}[(a)]
    \item $A \cap B$
    \item $B \cap C$
    \item $A \cup B$
    \item $A \cap \left( B \cup C \right)$
  \end{enumerate}

  \answer
  \begin{enumerate}[(a)]
    \item $A \cap B = \left\{ 2 \right\}$
    \item $B \cap C = \left\{ 5 \right\}$
    \item $\begin{array}{rcl}
      A \cup B
      & = & \left\{ x : x \in \mathds{N} \operatorname{and}
                        x \;\text{is even or prime} \right\} \\
      & = & \mathds{N} \setminus
            \left\{ x : x \;\text{is odd} \operatorname{and}
                        x \;\text{isn't prime} \right\} \\
    \end{array}$
    \item $\begin{array}{rcl}
      A \cap \left( B \cup C \right)
      & = & \left( A \cap B \cup \right) \left( A \cap C \right) \\
      & = & \left\{ 2 \right\} \cup
            \left\{ x : x \in \mathds{N} \operatorname{and}
                        x \;\text{is a multiple of both}\; 2
                         \;\text{and}\; 5 \right\} \\
      & = & \left\{ x : x \in \mathds{N} \operatorname{and}
                        x \;\text{is either}\; 2
                          \;\text{or a multiple of}\; 10 \right\} \\
    \end{array}$
  \end{enumerate}

\end{exercise}


\begin{exercise}{2}
  If $A = \left\{ a, b, c \right\}$,
     $B = \left\{ 1, 2, 3 \right\}$,
     $C = \left\{ x \right\}$, and
     $D = \emptyset$,
  list all of the elements in each of the following sets.
  \begin{enumerate}[(a)]
    \item $A \times B$
    \item $B \times A$
    \item $A \times B \times C$
    \item $A \times D$
  \end{enumerate}

  \answer
  \begin{enumerate}[(a)]
    \item $A \times B =
           \left\{(a, 1), (a, 2), (a, 3),
                  (b, 1), (b, 2), (b, 3),
                  (c, 1), (c, 2), (c, 3)\right\}$
    \item $B \times A =
           \left\{(1, a), (1, b), (1, c),
                  (2, a), (2, b), (2, c),
                  (3, a), (3, b), (3, c)\right\}$
    \item $A \times B \times C =
           \left\{(a, 1, x), (a, 2, x), (a, 3, x),
                  (b, 1, x), (b, 2, x), (b, 3, x),
                  (c, 1, x), (c, 2, x), (c, 3, x)\right\}$
    \item $\emptyset$
  \end{enumerate}

\end{exercise}


\begin{exercise}{3}
  Find an example of two nonempty sets $A$ and $B$ for which
  $A \times B = B \times A$ is true.
  \answer
  $A$ and $B$ should be the same,
  else the pairs $(a, b) \in A \times B$
  could never match the pairs $(b, a) \in B \times A$
  (i.e., we must have
   $\forall_{b \in B} \; b \in A$, and
   $\forall_{a \in A} \; a \in B$
   in order to have $(b, a) \in A \times B$
   for all $(b, a) \in B \times A$).
  As a constructive example, let $A = B = \left\{1\right\}$,
  so $A \times B = B \times A = \left\{(1, 1)\right\}$.
  It's worth noting that if $A \ne B$
  either $A$ or $B$ should be the empty set:
  let $b \in B$ but $b \notin A$,
  if $a \in A$, then $a \ne b$ and $(b, a) \in B \times A$,
  however $(b, a) \notin A \times B$,
  therefore $B \times A \ne A \times B$.
  As the empty set wouldn't be a solution,
  the only solution is $A = B$ for every nonempty set $A$.
\end{exercise}


\begin{exercise}{4}
  Prove $A \cup \emptyset = A$ and $A \cap \emptyset = \emptyset$.
  \answer
  Knowing that the logical $\lor$/$\land$ operators
  (respectively \emph{or/and}) satisfy:
  \begin{itemize}
    \item $p \operatorname{or} \text{False} = p \lor \text{False} = p$
    \item $p \operatorname{and} \text{False} = p \land \text{False}
                                             = \text{False}$
  \end{itemize}

  We can straightly apply the definition of union and intersection:
  \[
    A \cup \emptyset
    = \left\{ x : x \in A \operatorname{or}
                  \overbrace{\cancel{x \in \emptyset}}^{\text{False}}
      \right\}
    = \left\{ x : x \in A \right\}
    = A
  \]
  \[
    A \cap \emptyset
    = \left\{ x : x \in A \operatorname{and}
                  \overbrace{\cancel{x \in \emptyset}}^{\text{False}}
      \right\}
    = \left\{ x : False\right\}
    = \emptyset
  \]
  \qed
\end{exercise}


\begin{exercise}{5}
  Prove $A \cup B = B \cup A$ and $A \cap B = B \cap A$.
  \answer
  Since the logical \emph{or/and} operators are commutative,
  \[
    A \cup B
    = \left\{ x : x \in A \operatorname{or} x \in B\right\}
    = \left\{ x : x \in B \operatorname{or} x \in A\right\}
    = B \cup A
  \]
  \[
    A \cap B
    = \left\{ x : x \in A \operatorname{and} x \in B\right\}
    = \left\{ x : x \in B \operatorname{and} x \in A\right\}
    = B \cap A
  \]
  \qed
\end{exercise}


\begin{exercise}{6}
  Prove $A \cup (B \cap C) = (A \cup B) \cap (A \cup C)$.
  \answer
  \[\begin{array}{rcl}
    A \cup (B \cap C)
    &=& \left\{ x : x \in A \lor x \in (B \cap C) \right\} \\
    &=& \left\{ x : x \in A \lor
                    x \in \left\{ y : y \in B \land y \in C \right\}
        \right\} \\
    &=& \left\{ x : x \in A \lor
                    \left( x \in B \land x \in C \right)
        \right\} \\
    &=& \left\{ x : \left( x \in A \lor x \in B \right) \land
                    \left( x \in A \lor x \in C \right)
        \right\} \\
    &=& \left\{ x : x \in \left\{ y : y \in A \lor y \in B \right\}
                    \land
                    x \in \left\{ y : y \in A \lor y \in C \right\}
        \right\} \\
    &=& \left\{ x : x \in (A \cup B) \land x \in(A \cup C) \right\} \\
    &=& (A \cup B) \cap (A \cup C) \\
  \end{array}\]
  \qed
\end{exercise}


\begin{exercise}{7}
  Prove $A \cap (B \cup C) = (A \cap B) \cup (A \cap C)$.
  \[\begin{array}{rcl}
    A \cap (B \cup C)
    &=& \left\{ x : x \in A \land x \in (B \cup C) \right\} \\
    &=& \left\{ x : x \in A \land
                    x \in \left\{ y : y \in B \lor y \in C \right\}
        \right\} \\
    &=& \left\{ x : x \in A \land
                    \left( x \in B \lor x \in C \right)
        \right\} \\
    &=& \left\{ x : \left( x \in A \land x \in B \right) \lor
                    \left( x \in A \land x \in C \right)
        \right\} \\
    &=& \left\{ x : x \in \left\{ y : y \in A \land y \in B \right\}
                    \lor
                    x \in \left\{ y : y \in A \land y \in C \right\}
        \right\} \\
    &=& \left\{ x : x \in (A \cap B) \lor x \in(A \cap C) \right\} \\
    &=& (A \cap B) \cup (A \cap C) \\
  \end{array}\]
  \qed
\end{exercise}


\begin{exercise}{8}
  Prove $A \subset B$ if and only if $A \cap B = A$.
  \answer
  Let $A \subset B$, then $\forall_{a \in A} \; a \in B$,
  i.e., $(a \in A) \implies (a \in B)$.
  With that assumption, we can say,
  by absorption (conjunction introduction),
  that $(a \in A) \implies (a \in A) \land (a \in B)$,
  whereas, by simplification (conjunction elimination),
  we always have $(a \in A) \land (a \in B) \implies (a \in A)$.
  We can join these statements into a single one:
  $(a \in A) \iff (a \in A) \land (a \in B)$.
  Finally:
  \[
    A \cap B
    = \left\{ x : x \in A \land x \in B \right\}
    = \left\{ x : x \in A \right\}
    = A
  \]
  On the other hand, let $A \cap B = A$.
  That's to say that $(x \in A) \land (x \in B) \iff (x \in A)$,
  which entails $(x \in A) \implies (x \in A) \land (x \in B)$,
  whose right hand side can be simplified (by conjuntion elimination),
  yielding $(x \in A) \implies (x \in B)$.
  But that can be written as $\forall_{x \in A} \; x \in B$,
  i.e., $A \subset B$.
  \qed
\end{exercise}


\begin{exercise}{9}
  Prove $(A \cap B)' = A' \cup B'$.
  \answer
  This De Morgan's law regarding sets can be proved
  from a De Morgan's law regarding logical propositions.
  The De Morgan's laws in logic, are:
  \begin{enumerate}[(1)]
    \item $\lnot p \lor \lnot q = \lnot (p \land q)$
    \item $\lnot p \land \lnot q = \lnot (p \lor q)$
  \end{enumerate}
  Truth table (accordingly to classical logic)
  including both sides of the above tautologies:
  \begin{center}
    \begin{tabular}{rcccc}
      $p$                     & False & False & True  & True  \\
      $q$                     & False & True  & False & True  \\
      $p \land q$             & False & False & False & True  \\
      $p \lor q$              & False & True  & True  & True  \\
      $\lnot p$               & True  & True  & False & False \\
      $\lnot q$               & True  & False & True  & False \\
      $\lnot p \lor \lnot q$  & True  & True  & True  & False \\
      $\lnot (p \land q)$     & True  & True  & True  & False \\
      $\lnot p \land \lnot q$ & True  & False & False & False \\
      $\lnot (p \lor q)$      & True  & False & False & False \\
    \end{tabular}
  \end{center}
  Then, for some universal set $X$
  satisfying $A \subset X$ and $B \subset X$
  from which we can define the complement of a set:
  \[\begin{array}{rcl}
    A' \cup B'
    &=& \left\{ x : x \in A' \lor x \in B' \right\} \\
    &=& \left\{ x : (x \in X \land x \notin A) \lor
                    (x \in X \land x \notin B)
        \right\} \\
    &=& \left\{ x : x \in X \land
                    (x \notin A \lor x \notin B)
        \right\} \\
    &=& \left\{ x : x \in X \land
                    \left[ \lnot (x \in A) \lor \lnot (x \in B) \right]
        \right\} \\
    &\stackrel{\text{(1)}}{=}&
        \left\{ x : x \in X \land
                    \lnot \left[ x \in A \land x \in B \right]
        \right\} \\
    &=& \left\{ x : x \in X \land
                    \lnot (x \in A \cap B)
        \right\} \\
    &=& (A \cap B)'
  \end{array}\]
  \qed

  Another approach to prove this would be to split the equalness
  in two parts:
  \begin{itemize}
    \item $(A \cap B)' \subset A' \cup B'$ \\
      Let $x \in (A \cap B)'$, so $x \in X$ but $x \notin A \cap B$.
      Suppose $x \in A$, then $x \notin B$, since $x \notin A \cap B$.
      Conversely, if $x \notin A$, the $x \notin A \cap B$
      restriction is already satisfied.
      Therefore, either $x \notin B$ or $x \notin A$,
      which can be stated as $x \in A' \lor x\in B'$,
      i.e., $x \in A' \cup B'$.
      Therefore, $(A \cap B)' \subset A' \cup B'$.
    \item $A' \cup B' \subset (A \cap B)'$ \\
      Let $x \in A'$, so $x \in X$ but $x \notin A$.
      Since $A \cap B \subset A$,
      it's worth noting that $x \notin A \cap B$
      (if a value isn't in a set, it can't be in its subset).
      Then we have $x \in (A \cap B)'$.
      Likewise, let $y \in B'$,
      then due to the symmetry we can say that $y \in (A \cap B)'$.
      Then for any value in either $A'$ or $B'$,
      we know it's also in $(A \cap B)'$.
      That's to say $A' \cup B' \subset (A \cap B)'$.
  \end{itemize}
  \qed
\end{exercise}


\begin{exercise}{10}
  Prove $A \cup B = (A \cap B) \cup (A \setminus B)
                               \cup (B \setminus A)$.
  \answer
  For some universal set $X$,
  required in order to define set complements,
  we have:
  \[\begin{array}{rclr}
    (A \cap B) \cup (A \setminus B) \cup (B \setminus A)
    &=& \left[ (A \cap B) \cup (A \cap B) \right] \cup
        (A \setminus B) \cup (B \setminus A)
      & \text{(duplication)} \\
    &=& (A \cap B) \cup \left[ (A \cap B) \cup
        (A \setminus B) \right] \cup (B \setminus A)
      & \text{(associative)} \\
    &=& \left[ (A \cap B) \cup (A \setminus B) \right] \cup
        (A \cap B) \cup (B \setminus A)
      & \text{(commutative)} \\
    &=& \left[ (A \cap B) \cup (A \setminus B) \right] \cup
        \left[ (B \cap A) \cup (B \setminus A) \right]
      & \text{(associative + commutative)} \\
    &=& \left[ (A \cap B) \cup (A \cap B') \right] \cup
        \left[ (B \cap A) \cup (B \cap A') \right]
      & \text{(difference definition)} \\
    &=& \left[ A \cap (B \cup B') \right] \cup
        \left[ B \cap (A \cup A') \right]
      & \text{(distributive)} \\
    &=& \left[ A \cap X \right] \cup
        \left[ B \cap X \right] \\
    &=& \left[ A \right] \cup \left[ B \right] \\
    &=& A \cup B
  \end{array}\]
  \qed
\end{exercise}


\begin{exercise}{11}
  Prove $(A \cup B) \times C = (A \times C) \cup (B \times C)$.
  \answer
  It follows from the distributive property
  $(p \lor q) \land r = (p \land r) \lor (q \land r)$
  of the \emph{or/and} operators.
  \[\begin{array}{rcl}
    (A \cup B) \times C
    &=& \left\{ (x, y) : x \in A \cup B \land y \in C \right\} \\
    &=& \left\{ (x, y) : x \in \left\{ z : z \in A \lor z \in B
                               \right\} \land y \in C
        \right\} \\
    &=& \left\{ (x, y) : (x \in A \lor x \in B) \land y \in C
        \right\} \\
    &=& \left\{ (x, y) : (x \in A \land y \in C) \lor
                         (x \in B \land y \in C)
        \right\} \\
    &=& \left\{ (x, y) : x \in A \land y \in C \right\} \cup
        \left\{ (x, y) : x \in B \land y \in C \right\} \\
    &=& (A \times C) \cup (B \times C)
  \end{array}\]
  \qed
\end{exercise}


\begin{exercise}{12}
  Prove $(A \cap B) \setminus B = \emptyset$.
  \answer
  \[
    (A \cap B) \setminus B
    = (A \cap B) \cap B'
    = A \cap (B \cap B')
    = A \cap \emptyset
    = \emptyset
  \]
  \qed
\end{exercise}


\begin{exercise}{13}
  Prove $(A \cup B) \setminus B = A \setminus B$.
  \answer
  \[
    (A \cup B) \setminus B
    = (A \cup B) \cap B'
    = (A \cap B') \cup (B \cap B')
    = (A \cap B') \cup \emptyset
    = A \cap B'
    = A \setminus B
  \]
  \qed
\end{exercise}


\begin{exercise}{14}
  Prove $A \setminus (B \cup C) = (A \setminus B) \cap
                                  (A \setminus C)$.
  \answer
  \[\begin{array}{rcl}
    A \setminus (B \cup C)
    &=& A \cap (B \cup C)' \\
    &=& A \cap (B' \cap C') \\
    &=& (A \cap A) \cap (B' \cap C') \\
    &=& A \cap (A \cap B') \cap C' \\
    &=& (A \cap B') \cap A \cap C' \\
    &=& (A \setminus B) \cap (A \setminus C) \\
  \end{array}\]
  \qed
\end{exercise}


\begin{exercise}{15}
  Prove $A \cap (B \setminus C) = (A \cap B) \setminus (A \cap C)$.
  \answer
  \[\begin{array}{rcl}
    (A \cap B) \setminus (A \cap C)
    &=& (A \cap B) \cap (A \cap C)' \\
    &=& (A \cap B) \cap (A' \cup C') \\
    &=& \left[ (A \cap A) \cap B \right] \cap (A' \cup C') \\
    &=& \left[ A \cap (A \cap B) \right] \cap (A' \cup C') \\
    &=& \left[ (A \cap B) \cap A \right] \cap (A' \cup C') \\
    &=& (A \cap B) \cap \left[ A \cap (A' \cup C') \right] \\
    &=& (A \cap B) \cap \left[ (A \cap A') \cup (A \cap C') \right] \\
    &=& (A \cap B) \cap \left[ \emptyset \cup (A \cap C') \right] \\
    &=& (A \cap B) \cap (A \cap C') \\
    &=& A \cap (B \cap A) \cap C' \\
    &=& A \cap (A \cap B) \cap C' \\
    &=& (A \cap A) \cap B \cap C' \\
    &=& A \cap B \cap C' \\
    &=& A \cap (B \setminus C) \\
  \end{array}\]
  \qed
\end{exercise}


\begin{exercise}{16}
  Prove $(A \setminus B) \cup (B \setminus A) =
         (A \cup B) \setminus (A \cap B)$.
  \answer
  Let $X$ be the universal set, then:
  \[\begin{array}{rcl}
    (A \setminus B) \cup (B \setminus A)
    &=& (A \cap B') \cup (B \cap A') \\
    &=& \left[ (A \cap B') \cup B \right] \cap
        \left[ (A \cap B') \cup A' \right] \\
    &=& \left[ (A \cup B) \cap (B' \cup B) \right] \cap
        \left[ (A \cup A') \cap (B' \cup A') \right] \\
    &=& \left[ (A \cup B) \cap X \right] \cap
        \left[ X \cap (B' \cup A') \right] \\
    &=& \left[ (A \cup B) \right] \cap
        \left[ (B' \cup A') \right] \\
    &=& (A \cup B) \cap (A' \cup B') \\
    &=& (A \cup B) \cap (A \cap B)' \\
    &=& (A \cup B) \setminus (A \cap B) \\
  \end{array}\]
  \qed
\end{exercise}


\begin{exercise}{17}
  Which of the following relations $f : \mathds{Q} \to \mathds{Q}$
  define a mapping?
  In each case, supply a reason why $f$ is or is not a mapping.
  \begin{enumerate}[(a)]
    \item $f(p/q) = \frac{p+1}{p-2}$
    \item $f(p/q) = \frac{3p}{3q}$
    \item $f(p/q) = \frac{p+q}{q^2}$
    \item $f(p/q) = \frac{3p^2}{7q^2} - \frac{p}{q}$
  \end{enumerate}
  \answer
  \begin{enumerate}[(a)]
    \item
      It's not a mapping, since we have $f(1/2) = -2$,
      $f(2/4)$ doesn't exist, and $f(3/6) = 4$.
      As $1/2 = 2/4 = 3/6$, we're dealing with a single number,
      and this input would have had been mapped to a single outcome
      if $f$ were a mapping.
    \item
      It's $f(x) = 3x$, which happens to be a mapping,
      since it maps every single input $x \in \mathds{Q}$
      to a single outcome in $\mathds{Q}$.
    \item $f(p/q) = \frac{p+q}{q^2}$
      It's not a mapping, since we have $f(1/2) = 3/4$,
      and $f(2/4) = 6/16 = 3/8 \ne 3/4$,
      that is,
      it's a relation where a single input from $\mathds{Q}$
      yields distinct outcomes from $\mathds{Q}$
      depending on the fractional representation of the input.
    \item
      We can write it as $f(x) = \frac{3}{7}x^2 - x$,
      which maps every single input $x \in \mathds{Q}$
      to a single outcome in $\mathds{Q}$.
      That relation is a mapping.
  \end{enumerate}
\end{exercise}


\begin{exercise}{18}
  Determine which of the following functions are one-to-one
  and which are onto.
  If the function is not onto, determine its range.
  \begin{enumerate}[(a)]
    \item $f : \mathds{R} \to \mathds{R}$ defined by $f(x) = e^x$
    \item $f : \mathds{Z} \to \mathds{Z}$ defined by $f(n) = n^2 + 3$
    \item $f : \mathds{R} \to \mathds{R}$ defined by $f(x) = \sin x$
    \item $f : \mathds{Z} \to \mathds{Z}$ defined by $f(x) = x^2$
  \end{enumerate}
  \answer
  The \emph{codomains} are always given,
  and the word ``\emph{range}'' in this book
  always stands for the \emph{image} of a mapping.
  I'm assuming $0 \in \mathds{N}$
  and $\mathds{N}^* = \mathds{N} \setminus \{0\}$.
  None of these functions is surjective/onto,
  their \emph{ranges/images} are:
  \begin{enumerate}[(a)]
    \item $\left\{ x \in \mathds{R} : x > 0 \right\}$
    \item $\left\{ x \in \mathds{N} :
                     \sqrt{x - 3} \in \mathds{N} \right\}$
    \item $\left\{ x \in \mathds{R} : -1 \le x \le 1 \right\}$
    \item $\left\{ x \in \mathds{N} :
                     \sqrt{x} \in \mathds{N} \right\}$
  \end{enumerate}
  Only one of these examples is injective/one-to-one:
  \begin{enumerate}[(a)]
    \item One-to-one,
          since $e^x$ is a strictly increasing monotonic function
          (i.e., $x > y \implies e^x > e^y$);
    \item Not one-to-one,
          since $\forall_{n \in \mathds{N}^*} \; f(-n) = f(n)$;
    \item Not one-to-one,
          since $\forall_{k \in \mathds{Z}, x \in \mathds{R}} \;
                 f(x + k 2 \pi) = f(x)$;
    \item Not one-to-one,
          since $\forall_{n \in \mathds{N}^*} \; f(-n) = f(n)$.
  \end{enumerate}
\end{exercise}


\begin{exercise}{19}
  Let $f : A \to B$ and $g : B \to C$ be invertible mappings;
  that is, mappings such that $f^{-1}$ and $g^{-1}$ exist.
  Show that $(g \circ f)^{-1} = f^{-1} \circ g^{-1}$.
  \answer
  Since $f$ and $g$ are invertible,
  they're bijective.
  Then, for all values of $a$,
  we have a single value $b = f(a)$
  and a single value $c = g(b)$.
  Their inverses are $a = f^{-1}(b)$ and $b = g^{-1}(c)$,
  whereas the composition is:
  \[(g \circ f)(a) = g(f(a)) = g(b) = c\]
  Whose inverse is:
  \[
    (g \circ f)^{-1}(c)
    = a
    = f^{-1}(b)
    = f^{-1}(g^{-1}(c))
    = (f^{-1} \circ g^{-1})(c)
  \]
  \qed
\end{exercise}


\begin{exercise}{20}
  \begin{enumerate}[(a)]
    \item Define a function $f : \mathds{N} \to \mathds{N}$
          that is one-to-one but not onto.
    \item Define a function $f : \mathds{N} \to \mathds{N}$
          that is onto but not one-to-one.
  \end{enumerate}
  \answer
  \begin{enumerate}[(a)]
    \item $f(n) = n^2$ \\
      It's one-to-one (a strictly increasing monotonic function),
      yet not onto (e.g. since $\pm \sqrt{3} \notin \mathds{N}$,
                         $n = 3$ isn't in the range).
    \item $f(n) = \lfloor n / 2 \rfloor$ \\
      It's onto, since we can get every number $k \in \mathds{N}$
      by applying $n = 2 k$ as the input
      (i.e., the image and codomain are the same set),
      but it's not one-to-one (e.g. both inputs $n = 2$ and $n = 3$
                                    yields $f(n) = 1$).
  \end{enumerate}
\end{exercise}


\begin{exercise}{21}
  Prove the relation defined on $\mathds{R}^2$ by
  $(x_1, y_1) \sim (x_2, y_2)$ if $x_1^2 + y_1^2 = x_2^2 + y_2^2$
  is an equivalence relation.
  \answer
  To be an equivalence relation, it must be
  reflexive, symmetric and transitive.
  Actually, that's pretty straightforward
  since these are properties of the equalness operator.
  \begin{itemize}
    \item $(x, y) \sim (x, y)$ (reflexive) \\
      Since $x^2 + y^2 = x^2 + y^2$ always holds,
      it's a reflexive relation.
    \item $(x_1, y_1) \sim (x_2, y_2) \implies
           (x_2, y_2) \sim (x_1, y_1)$ (symmetric) \\
      Given $(x_1, y_1) \sim (x_2, y_2)$,
      we have $x_1^2 + y_1^2 = x_2^2 + y_2^2$,
      from which we get $x_2^2 + y_2^2 = x_1^2 + y_1^2$,
      meaning $(x_2, y_2) \sim (x_1, y_1)$.
    \item $(x_1, y_1) \sim (x_2, y_2) \land (x_2, y_2) \sim (x_3, y_3)
           \implies (x_1, y_1) \sim (x_3, y_3)$ (transitive) \\
      Say $x_1^2 + y_1^2 = z$.
      From $(x_1, y_1) \sim (x_2, y_2)$,
      we get $x_1^2 + y_1^2 = x_2^2 + y_2^2$,
      so $z = x_2^2 + y_2^2$.
      From $(x_2, y_2) \sim (x_3, y_3)$,
      we have $x_2^2 + y_2^2 = x_3^2 + y_3^2$,
      so $z = x_3^2 + y_3^2$.
      Since $z$ is always the same (a squared radius),
      we have $x_1^2 + y_1^2 = x_3^2 + y_3^2$,
      that is, $(x_1, y_1) \sim (x_3, y_3)$.
  \end{itemize}
  \qed
\end{exercise}


\begin{exercise}{22}
  Let $f : A \to B$ and $g : B \to C$ be maps.
  \begin{enumerate}[(a)]
    \item If $f$ and $g$ are both one-to-one functions,
          show that $g \circ f$ is one-to-one.
    \item If $g \circ f$ is onto, show that $g$ is onto.
    \item If $g \circ f$ is one-to-one, show that $f$ is one-to-one.
    \item If $g \circ f$ is one-to-one and $f$ is onto,
          show that $g$ is one-to-one.
    \item If $g \circ f$ is onto and $g$ is one-to-one,
          show that $f$ is onto.
  \end{enumerate}
  \answer
  \begin{enumerate}[(a)]
    \item
      Let $(g \circ f)(x) = (g \circ f)(y) = c$
      for some $x \in A$, $y \in A$ and $c \in C$.
      That is, $g(f(x)) = g(f(y)) = c$.
      Since $g$ is one-to-one (injective),
      $f(x) = f(y) = b$, where $b \in B$.
      Since $f$ in also one-to-one, it follows that $x = y$,
      then for each $c$ in the image of $g \circ f$,
      there's only a single $x \in A$ such that
      $(g \circ f)(x) = c$.
      \qed
    \item
      Since $g \circ f$ is onto (surjective),
      there's always an $a \in A$
      such that $(g \circ f)(a) = c$
      for any $c \in C$.
      For a given $a$, let $f(a) = b$,
      then we can write $c = (g \circ f)(a) = g(f(a)) = g(b)$.
      That is, for any $c \in C$, there's always a $b \in B$
      such that $g(b) = c$, namely $b = f(a)$.
      \qed
    \item
      Let $f(x) = f(y) = b$
      for some $x \in A$, $y \in A$ and $b \in B$,
      and let $g(b) = c$ for some $c \in C$.
      Then we can write $g(f(x)) = g(f(y)) = g(b) = c$.
      That's to say that $(g \circ f)(x) = (g \circ f)(y) = c$,
      but since $g \circ f$ is one-to-one, we must have $x = y$.
      Then for each $b$ in the image of $f$,
      there's only a single $x \in A$ such that $f(x) = b$.
      \qed
    \item
      Suppose $g$ isn't one-to-one, then for some $c \in C$
      there exists distinct $x \in B$ and $y \in B$
      such that $g(x) = g(y) = c$.
      Since $f$ is onto, there's always a $v \in A$ and a $w \in A$
      such that $f(v) = x$ and $f(w) = y$.
      However, $(g \circ f)(v) = g(f(v)) = g(x) = c$
      and $(g \circ f)(w) = g(f(w)) = g(y) = c$,
      and since $g \circ f$ is one-to-one, $v = w$,
      which also means that $x = y$, leading to a contradiction.
      Then our assumption must be false, and $g$ must be one-to-one.
      \qed
    \item
      Suppose $f$ isn't onto,
      then $\exists_{b \in B} \forall_{a \in A} \; f(a) \ne b$.
      Let $g(b) = c$ for some $c \in C$.
      Since $g \circ f$ is onto,
      $\exists_{a \in A} \; (g \circ f)(a) = c$.
      However, $g(b) = c = (g \circ f)(a) = g(f(a))$,
      and since $g$ is one-to-one, we must have $b = f(a)$.
      Therefore our assumption is false, and $f$ must be onto.
      \qed
  \end{enumerate}
\end{exercise}


\begin{exercise}{23}
  Define a function on the real numbers by
  \[f(x) = \frac{x+1}{x-1}\]
  What are the domain and range of $f$?
  What is the inverse of $f$?
  Compute $f \circ f^{-1}$ and $f^{-1} \circ f$.
  \answer
  By definition, it's a function on the real numbers,
  but we can't evaluate this function where the denominator is zero,
  i.e., the domain is $\mathds{R} \setminus \{1\}$.
  Let's try to find the $x$ from a given $f(x)$ value,
  assuming $x \ne 1$:
  \[\begin{array}{rrcl}
               & (x - 1) f(x) &=& x + 1 \\
               & x f(x) - f(x) &=& x + 1 \\
               & x f(x) - x &=& f(x) + 1 \\
               & x (f(x) - 1) &=& f(x) + 1 \\
    \therefore & x &=& \frac{f(x) + 1}{f(x) - 1} \\
  \end{array}\]
  In the final part we needed to assume that $f(x) \ne 1$,
  since there's no $x$ such that $x \cdot 0 = 2$,
  which means there's no $x$ such that $f(x) = 1$.
  The codomain is $\mathds{R}$,
  yet the image/range is $\mathds{R} \setminus \{1\}$.
  Surprisingly enough, we've found that $f = f^{-1}$.
  Therefore, $f \circ f^{-1} = f^{-1} \circ f = f \circ f$,
  and, as expected (assuming $x \ne 1$):
  \[
    (f \circ f)(x)
    = \frac{\frac{x+1}{x-1}+1}{\frac{x+1}{x-1}-1}
    = \frac{\frac{(x+\cancel{1})+(x-\cancel{1})}{x-1}}
           {\frac{(\cancel{x}+1)-(\cancel{x}-1)}{x-1}}
    = \frac{\frac{2 x}{x-1}}{\frac{2}{x-1}}
    = \frac{\cancel{2} x}{\cancel{x-1}}
      \frac{\cancel{x-1}}{\cancel{2}}
    = x
  \]
\end{exercise}


\begin{exercise}{24}
  Let $f : X \to Y$ be a map
  with $A_1, A_2 \subset X$ and $B_1, B_2 \subset Y$.
  \begin{enumerate}[(a)]
    \item Prove $f(A_1 \cup A_2) = f(A_1) \cup f(A_2)$.
    \item Prove $f(A_1 \cap A_2) \subset f(A_1) \cap f(A_2)$.
          Give an example in which equality fails.
    \item Prove $f^{-1}(B_1 \cup B_2) = f^{-1}(B_1) \cup f^{-1}(B_2)$,
          where \[f^{-1}(B) = \left\{ x \in X : f(x) \in B \right\}.\]
    \item Prove $f^{-1}(B_1 \cap B_2) = f^{-1}(B_1) \cap f^{-1}(B_2)$.
    \item Prove $f^{-1}(Y \setminus B_1) = X \setminus f^{-1}(B_1)$.
  \end{enumerate}
  \answer
  \begin{enumerate}[(a)]

    \item
      \[\begin{array}{rcl}
        f(A_1 \cup A_2)
        &=& \left\{ f(a) : a \in A_1 \cup A_2 \right\} \\
        &=& \left\{ f(a) : a \in \left\{ x : x \in A_1 \lor x \in A_2
                                 \right\}
            \right\} \\
        &=& \left\{ f(a) : a \in A_1 \lor a \in A_2 \right\} \\
        &=& \left\{ f(a) : a \in A_1 \right\} \cup
            \left\{ f(a) : a \in A_2 \right\} \\
        &=& f(A_1) \cup f(A_2)
      \end{array}\]
      We can always split the comprehension
      on the ``$\lor$'' operator
      as the union of two sets
      because the constraints are independent.
      \qed
      The same can been proven
      by splitting the equalness in two parts
      (the former being both a subset and a superset of the latter)
      and proving them separately:
      \begin{itemize}
        \item
          For all $b_1 \in f(A_1)$,
          we have some $a_1 \in A_1$
          such that $b_1 = f(a_1)$.
          But we can increase the set
          and say that $a_1 \in A_1 \cup A_2$,
          which means that $f(a_1) \in f(A_1 \cup A_2)$.
          Therefore, $f(A_1) \subset f(A_1 \cup A_2)$.
          Likewise, $f(A_2) \subset f(A_1 \cup A_2)$,
          so $f(A_1) \cup f(A_2) \subset f(A_1 \cup A_2)$.
        \item
          On the other hand, let $b \in f(A_1 \cup A_2)$,
          then $b = f(a)$ for some $a \in A_1 \cup A_2$.
          Either $a \in A_1$ or $a \notin A_1$.
          If $a \in A_1$, then $f(a) \in f(A_1)$,
          else $a \in A_2$ because it's in the union of these sets,
          but in this case $f(a) \in f(A_2)$.
          That is, $f(a) \in f(A_1) \lor f(a) \in f(A_2)$,
          which can be written as $f(a) \in f(A_1) \cup f(A_2)$,
          or simply as $f(A_1 \cup A_2) \subset f(A_1) \cup f(A_2)$.
      \end{itemize}
      \qed

    \item
      Let $b \in f(A_1 \cap A_2)$,
      then $\exists_{a \in A_1 \cap A_2} \; f(a) = b$.
      From the definition of intersection,
      we know that $a \in A_1$, and that $a \in A_2$.
      Therefore, $f(a) \in f(A_1)$ and $f(a) \in f(A_2)$,
      yielding $b \in f(A_1) \land b \in f(A_2)$,
      that is, $b \in f(A_1) \cap f(A_2)$.
      Since that's the same for any such $b$,
      we've found that $f(A_1 \cap A_2) \subset f(A_1) \cap f(A_2)$.
      \qed
      As an example where the reverse doesn't hold,
      let $A_1 = \{0, 1, 2\}$, $A_2 = \{0, -1, -2\}$
      and $f(a) = a^2$.
      Then $f(A_1) = \{0, 1, 4\} = f(A_2) = f(A_1) \cap f(A_2)$.
      However, $A_1 \cap A_2 = \{0\}$, then $f(A_1 \cap A_2) = \{0\}$.
      As expected, $f(A_1 \cap A_2) \subset f(A_1) \cap f(A_2)$,
      but in this case
      $f(A_1 \cap A_2) \not\supset f(A_1) \cap f(A_2)$.

    \item
      By splitting the set comprehension
      by its [independent] disjunction terms:
      \[\begin{array}{rcl}
        f^{-1}(B_1 \cup B_2)
        &=& \left\{ a : f(a) \in B_1 \cup B_2 \right\} \\
        &=& \left\{ a : f(a) \in \left\{ x : x \in B_1 \lor x \in B_2
                                 \right\}
            \right\} \\
        &=& \left\{ a : f(a) \in B_1 \lor f(a) \in B_2 \right\} \\
        &=& \left\{ a : f(a) \in B_1 \right\} \cup
            \left\{ a : f(a) \in B_2 \right\} \\
        &=& f^{-1}(B_1) \cup f^{-1}(B_2)
      \end{array}\]
      \qed
      Or by splitting the equalness in two parts:
      \begin{itemize}
        \item
          For all $a_1 \in f^{-1}(B_1)$,
          we have some $f(a_1) \in B_1$.
          But we can increase the latter set
          and say that $f(a_1) \in B_1 \cup B_2$,
          which means that $a_1 \in f^{-1}(B_1 \cup B_2)$.
          Therefore, $f^{-1}(B_1) \subset f^{-1}(B_1 \cup B_2)$.
          Likewise, $f^{-1}(B_2) \subset f^{-1}(B_1 \cup B_2)$, so
          $f^{-1}(B_1) \cup f^{-1}(B_2) \subset f^{-1}(B_1 \cup B_2)$.
        \item
          Let $a \in f^{-1}(B_1 \cup B_2)$,
          then there is some $f(a) \in B_1 \cup B_2$.
          Either $f(a) \in B_1$ or $f(a) \notin B_1$.
          If $f(a) \in B_1$, then $a \in f^{-1}(B_1)$,
          else $f(a) \in B_2$ since it's in the union of these sets,
          but in this case $a \in f^{-1}(B_2)$.
          That is, $a \in f^{-1}(B_1) \lor a \in f^{-1}(B_2)$,
          which can be written as $a \in f^{-1}(B_1) \cup f^{-1}(B_2)$.
          Finally, we can say that
          $f^{-1}(B_1 \cup B_2) \subset f^{-1}(B_1) \cup f^{-1}(B_2)$.
      \end{itemize}
      \qed

    \item
      Splitting the equalness:
      \begin{itemize}
        \item $f^{-1}(B_1 \cap B_2) \subset
               f^{-1}(B_1) \cap f^{-1}(B_2)$ \\
          Let $a \in f^{-1}(B_1 \cap B_2)$.
          From the definition of $f^{-1}$ applied on a set,
          it's clear that $f(a) \in B_1 \cap B_2$,
          which can be split as $f(a) \in B_1$ and $f(a) \in B_2$.
          Therefore, $a \in f^{-1}(B_1)$ and $a \in f^{-1}(B_2)$,
          which can be joined as
          $a \in f^{-1}(B_1) \cap f^{-1}(B_2)$.
        \item $f^{-1}(B_1 \cap B_2) \supset
               f^{-1}(B_1) \cap f^{-1}(B_2)$ \\
          Let $a \in f^{-1}(B_1) \cap f^{-1}(B_2)$.
          Then we can say that
          $a \in f^{-1}(B_1)$ and $a \in f^{-1}(B_2)$,
          which means that
          $f(a) \in B_1$ and $f(a) \in B_2$.
          Therefore, $f(a) \in B_1 \cap B_2$,
          which means that $a \in f^{-1}(B_1 \cap B_2)$.
      \end{itemize}
      \qed
      In this case, as an easier proof,
      we could have splitten the set comprehension in its conjunction,
      since it becomes the definition of an intersection of sets:
      \[\begin{array}{rcl}
        f^{-1}(B_1 \cap B_2)
        &=& \left\{ a : f(a) \in B_1 \cap B_2 \right\} \\
        &=& \left\{ a : f(a) \in \left\{ x : x \in B_1 \land x \in B_2
                                 \right\}
            \right\} \\
        &=& \left\{ a : f(a) \in B_1 \land f(a) \in B_2 \right\} \\
        &=& \left\{ a : a \in f^{-1}(B_1) \land
                        a \in f^{-1}(B_2) \right\} \\
        &=& f^{-1}(B_1) \cap f^{-1}(B_2)
      \end{array}\]
      \qed

    \item
      Since $Y$ is the codomain of $f$, $f(a) \in Y$ always holds.
      Likewise, since $X$ is the domain, we always have $a \in X$.
      From the straight application
      of the definition of $f$ and $f^{-1}$ on sets,
      we know that $f(X) \subset Y$ (the range is in the codomain),
      yet $f^{-1}(Y) = \left\{ a \in X : f(a) \in Y \right\} = X$
      since the predicate is always true.
      \[\begin{array}{rcl}
        f^{-1}(Y \setminus B_1)
        &=& \left\{ a : f(a) \in Y \setminus B_1 \right\} \\
        &=& \left\{ a : f(a) \in \left\{ x : x \in Y \land x \notin B_1
                                 \right\}
            \right\} \\
        &=& \left\{ a : f(a) \in Y \land
                        f(a) \notin B_1 \right\} \\
        &=& \left\{ a : f(a) \in Y \land
                        \lnot f(a) \in B_1 \right\} \\
        &=& \left\{ a : a \in f^{-1}(Y) \land
                        \lnot a \in f^{-1}(B_1) \right\} \\
        &=& \left\{ a : a \in X \land
                        a \notin f^{-1}(B_1) \right\} \\
        &=& X \setminus f^{-1}(B_1)
      \end{array}\]
      \qed
  \end{enumerate}
\end{exercise}


\begin{exercise}{25}
  Determine whether or not
  the following relations are equivalence relations
  on the given set.
  If the relation is an equivalence relation,
  describe the partition given by it.
  If the relation is not an equivalence relation,
  state why it fails to be one.
  \begin{enumerate}[(a)]
    \item $x \sim y$ in $\mathds{R}$ if $x \ge y$
    \item $m \sim n$ in $\mathds{Z}$ if $m n = 0$
    \item $x \sim y$ in $\mathds{R}$ if $|x - y| \le 4$
    \item $m \sim n$ in $\mathds{Z}$
          if $m \equiv n \;\; (\operatorname{mod} 6)$
  \end{enumerate}
\end{exercise}


\end{document}
